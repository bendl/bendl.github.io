%-------------------------
% Resume in Latex
% Author : Jake Gutierrez
% Based off of: 
% License : MIT
%------------------------

\documentclass[letterpaper,11pt]{article}

\usepackage{latexsym}
\usepackage[empty]{fullpage}
\usepackage{titlesec}
\usepackage{marvosym}
\usepackage[usenames,dvipsnames]{color}
\usepackage{verbatim}
\usepackage{enumitem}
\usepackage[colorlinks = true,
            linkcolor = blue,
            urlcolor  = blue,
            citecolor = blue,
            anchorcolor = blue]{hyperref}
\usepackage{fancyhdr}
\usepackage[english]{babel}
\usepackage{tabularx}
\usepackage{multicol}


%----------FONT OPTIONS----------
% sans-serif
%\usepackage[sfdefault]{FiraSans}
%\usepackage[sfdefault]{roboto}
%\usepackage[sfdefault]{noto-sans}
\usepackage{helvet}
%\renewcommand*\familydefault{\sfdefault}

%\usepackage[default]{sourcesanspro}

% serif
%\usepackage{charter}


\pagestyle{fancy}
\fancyhf{} % clear all header and footer fields
\fancyfoot{}
\renewcommand{\headrulewidth}{0pt}
\renewcommand{\footrulewidth}{0pt}

% Adjust margins
\addtolength{\oddsidemargin}{-0.5in}
\addtolength{\evensidemargin}{-0.5in}
\addtolength{\textwidth}{1in}
\addtolength{\topmargin}{-0.5in}
\addtolength{\textheight}{1.0in}

\urlstyle{same}

\raggedbottom
\raggedright
\setlength{\tabcolsep}{0in}
\setlist[itemize]{leftmargin=1.5cm}

% Sections formatting
\titleformat{\section}{
  \vspace{-15pt}\raggedright%\bfseries
}{}{0em}{}[\color{black}\titlerule \vspace{-2pt}]

% Ensure that generate pdf is machine readable/ATS parsable
\pdfgentounicode=1

%-------------------------
% Custom commands
\newcommand{\resumeItem}[1]{
  \item\small{
    {#1 \vspace{-5pt}}
  }
}

\newcommand{\resumeItemWithoutTitle}[1]{
  \item\small{
    {\vspace{-2pt}}
  }
}

%\begin{tabularx}{1\textwidth}{p{8cm}Xr}
%  \small\textbf{RTL Design Engineer} & \small\textbf{ARM Limited} & %\small\textbf{September 2019 -- present}
%\end{tabularx}\vspace{-3pt}

\newcommand{\resumeSubheading}[4]{
  \vspace{-1pt}
    \begin{tabular*}{\textwidth}{l@{\extracolsep{\fill}}r}
      \textbf{#1} & #2 \\
      \textit{#3} & \textit{#4} \\
    \end{tabular*}\vspace{-3pt}
}

\newcommand{\resumeSubItem}[2]{\resumeItem{#1}{#2}\vspace{-3pt}}

\renewcommand{\labelitemii}{$\circ$}

\newcommand{\resumeSubHeadingListStart}{\begin{itemize}[leftmargin=*, label={}]}
\newcommand{\resumeSubHeadingListEnd}{\end{itemize}}
\newcommand{\resumeItemListStart}{\begin{itemize}[leftmargin=10mm]}
\newcommand{\resumeItemListEnd}{\end{itemize}\vspace{-5pt}}

%-------------------------------------------
%%%%%%  RESUME STARTS HERE  %%%%%%%%%%%%%%%%%%%%%%%%%%%%


\begin{document}

%----------HEADING----------
% \begin{tabular*}{\textwidth}{l@{\extracolsep{\fill}}r}
%   \textbf{\href{http://sourabhbajaj.com/}{\Large Sourabh Bajaj}} & Email : \href{mailto:sourabh@sourabhbajaj.com}{sourabh@sourabhbajaj.com}\\
%   \href{http://sourabhbajaj.com/}{http://www.sourabhbajaj.com} & Mobile : +1-123-456-7890 \\
% \end{tabular*}

\iffalse
\begin{center}
    \textbf{\Huge \scshape Ben Lancaster} \\ \vspace{5pt}
    \small (+44) 07722 358258 $|$ \href{ben.lancaster.msc@gmail.com}{\underline{ben.lancaster.msc@gmail.com}} $|$ 
    \href{https://uk.linkedin.com/in/bendl}{\underline{uk.linkedin.com/in/bendl}} $|$
    \href{https://github.com/bendl}{\underline{github.com/bendl}}
\end{center}
\fi

%\iffalse
\begin{minipage}{4in}
    \textbf{\Huge Ben Lancaster}\\
    \href{mailto:ben.lancaster.msc@gmail.com}
    {ben.lancaster.msc@gmail.com}\\
    07722 358258
\end{minipage}
    \hfill
\begin{minipage}{3in}
\begin{flushright}
    \href{https://uk.linkedin.com/in/bendl}{https://uk.linkedin.com/in/bendl}\\
    \href{https://github.com/bendl}{https://github.com/bendl}  \\
    
\end{flushright}
\end{minipage}
\vspace{15pt}
%\fi

\iffalse
\section{SUMMARY}
Passionate about CPU design and saving clock cycles.
\vspace{15pt}
\fi

\section{EXPERIENCE}
    \resumeSubheading
    {Senior CPU Design Engineer}
    {2019 -- present}
    {Arm Ltd - Processor Design Group}
    {Cambridge, UK}
    \resumeItemListStart
    
\resumeItem{Micro-architecture specification and RTL design for A-class CPUs and multi-core clusters for client market targeting smartphones and large screen compute devices.}

\resumeItem{Responsible for power/clock/reset logic for upcoming A-class CPUs featuring multiple power/clock/reset domains and crossings. Experience with CDC and RDC analysis.}

\resumeItem{Extended CPU IP products to support ASIL B/D for automotive and safety critical applications (ISO 26262).}

\resumeItem{Unit design lead for Memory system filtering unit, MPAM, and RAS units for an upcoming System IP.}

%\resumeItem{Worked on micro-architecture specification and RTL design of a fully-coherent System Level Cache for an upcoming 5nm IP targeting high-end smartphones.}

\resumeItem{Contributions to design of a fully-coherent L3 System Level Cache using AMBA CHI and AXI5.}

\resumeItem{Created automation flows for generating synthesizable SystemVerilog RTL from machine-readable specifications.}

\resumeItem{Experience implementing AMBA interfaces including CHI, AXI, and APB.}

%\resumeItem{Designed and implemented a post-silicon debug feature for MMU-700's Translation Buffer and Translation Control Unit compatible with Arm CoreSight ELA-600.}
    \resumeItemListEnd
    \vspace{15pt}
    \resumeSubheading
        {Firmware Engineer, Internship}
        {2016 - 2017}
        {Spirent Communications - Positioning Group}
        {Paignton, UK}
    \resumeItemListStart
%\resumeItem{Worked on hardware design of next generation GNSS signal generators.}
\resumeItem{Using Xilinx Virtex UltraScale+/Spartan FPGAs for RF signal generation (GNSS/VHF/UHF).}
\resumeItem{Embedded C programming on Xilinx MicroBlaze and PIC16/24 micro-controllers.}
\resumeItem{Implemented software algorithm for dynamic power attenuation and calibration using programmable DAC for GNSS RF signal generators.}
\resumeItem{Controlling EEPROMs, LEDs, on-board fans, and other peripherals with I2C and SMBus.}
    \resumeItemListEnd

\vspace{10pt}
\section{EDUCATION}
    \resumeSubheading
        {MSc (Eng) Embedded Systems Engineering, 1st}
        {2018 -- 2019}
        {University of Leeds}
        {United Kingdom}
    \resumeItemListStart
        \resumeItem{1st, Final Project: Multi-core RISC SoC Design and Implementation for FPGAs.}
        \resumeItem{Courses include: FPGA Design for System-on-Chip, Digital Signal Processing for Communications, Embedded
Microprocessor System Design, Circuit Analysis, Electronics for Medical Devices}
    \resumeItemListEnd
    
     \vspace{10pt}
     \resumeSubheading
     {BSc (Hons) Computer Science, 1st}
     {2014 -- 2018}
     {University of Plymouth}
     {United Kingdom}
    \resumeItemListStart
        \resumeItem{Top Final Year Student, Best Final Project, Revell Research Systems Price}
        \resumeItem{Final Project: FPGA-based 16-bit RISC soft-microprocessor (with IO \& interrupts) and CFG Compiler}
        %\resumeItem{Courses: Digital Electronics, Embedded Systems and Compilers, Machine Vision, Parallel Computation}
    \resumeItemListEnd

\vspace{10pt}
\section{ADDITIONAL EXPERIENCE \& AWARDS}
    \vspace{-15pt}
\begin{multicols}{2}
    \resumeItemListStart
    \resumeItem{Best Final Project}
    \resumeItem{Top Final Year Student}
    \resumeItemListEnd
\columnbreak
    \resumeItemListStart
    \resumeItem{Dean’s List 2015-2018 member}
    \resumeItem{Revell Research Systems Prize}
    \resumeItemListEnd
\end{multicols}\vspace{5pt}

%-----------PROJECTS-----------
\iftrue
\section{OPEN-SOURCE PROJECTS \& CONTRIBUTIONS}
    \resumeItemListStart
    \resumeItem{Multi-core RISC CPU design. Fits up to 96 cores on Spartan-6 and Cyclone V FPGAs.}
    \resumeItem{16-bit RISC ISA and CPU design. Written in Verilog and C compiler for a C-like programming language.}
    \resumeItem{Custom PCB with an ST Cortex M0 and debugging header. A 2-layer board for the Minispartan6+ FPGA development kit. Features an STM32F0 TSSOP processor, dual power supplies, I2C, ICSP, and LEDs.}
\resumeItemListEnd
\fi

%-------------------------------------------
\end{document}
